\documentclass[12pt]{article}

\usepackage[margin=1in]{geometry}
\usepackage{graphicx, color}
\setlength{\topmargin}{-1in}
\linespread{1.5}
\usepackage[T1]{fontenc}
\usepackage{kpfonts,baskervald}
\usepackage{lipsum}

\title{How do we sort these goddamn robots: Label availability changes object classification}
\author{Cody Kommers, Maddie Snyder}
\date{December 2016}

\begin{document}
\maketitle

\noindent How do we sort objects into categories in the world? Typically, in the mind, objects are inextricably tied to their word labels. Seeing a tree evokes the word `tree', seeing your friend evokes their name, etc. 

\noindent Here we build a model of image categorization in which the categorization of objects is dependent on the labeled examples available during learning. If there is one labeled object presented, generalization of the label category will occur across objects that are most similar along the dimension of lowest visual variance. If there is more than one object labeled, then generalization will occur across the dimension of highest visual variance. 

\vspace{.5in}

\noindent Schema: 



\noindent Confounding factor: Subjects may judge ''fepness'' by the difference between the lengths of the head and the body. BUT this might actually be the dimension of lowest variance, visually. 

\end{document}